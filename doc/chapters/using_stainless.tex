\chapter{Using Stainless}
\label{chap:using_stainless}
This chapter describes the setup and integration process of Stainless in a new or already existing project.
It also shows the compatibility of Stainless with Scala as Stainless supports only a purely functional subset of Scala which they call \emph{Pure Scala}.

\section{Configuration (Set-up, sbt, jar, frontends/../stainless-scalac)}
There are two ways to integrate Stainless in a Scala project, the scala build tool (abt) plugin or the stainless.jar file.
Both, when run, analyses the passed code and report warnings to the console about the given code.
Stainless requires and Scala recommends Java SE Development Kit 8.
Newer Java versions won't compile.

\subsection{sbt}
sbt for Scala is like gradle or maven for Java.
It can compile Scala code continously or manual, manage dependencies with support for Maven-formatted repositories, mixing Scala and Java projects and much more.

A simple sbt project has the following structure:
\dirtree{%
  .1 ..
  .2 project.
  .3 build.properties.
  .3 plugins.sbt.
  .2 src.
  .3 main.
  .4 scala.
  .5 Main.scala.
  .2 build.sbt.
}
\emph{build.properties} specifies the abt version used for this project.
If the version is not available locally, sbt will download it.

In \emph{plugins.sbt} new sbt plugins can be added.
A plugin extends the build definition.
Mostly this means adding and overriding settings.

\emph{build.sbt} defines the build definition.
There can be several projects or subprojects as sbt doc calls it.

Here an example for a single project in \emph{build.sbt}:
\begin{lstlisting}[language=scala]
scalaVersion := "2.12.8"

lazy val root = (project in file("."))
\end{lstlisting}
The project is called root and its source files are located in the files directory.
Executing \code{sbt compile} should now compile the code.

The Stainless webpage has a guide on how to integrate Stainless in an existing project.
The simplifies steps are:
\begin{itemize}
  \item Install an external solver.
  \item Add Stainless sbt plugin to \emph{plugins.sbt}
  \item Enable the plugin in \emph{build.sbt} for the project.
\end{itemize}
After this setup, Stainless will report errors to the console, when running \code{sbt compile}.

\subsection{JAR file}


\section{Scala compatibility (Pure Scala, imperative features, dedicated BigInt, Generics...)}


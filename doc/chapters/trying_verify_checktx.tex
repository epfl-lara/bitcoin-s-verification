\chapter{Trying to Verify checkTransaction}
\label{chap:connecting}
// TODO

\section{Creation of a Transaction}
Some parts of the code in this section are from Bitcoin-S-Core transaction builder example.\cite{BitcoinSCore:txbuilderexample}
Bitcoin-S-Core has a bitcoin transaction builder class with the following signature:
\begin{lstlisting}[language=scala]
  BitcoinTxBuilder(
    destinations: Seq[TransactionOutput], // where we send money
    utxos: BitcoinTxBuilder.UTXOMap,      // unspent transaction outputs
    feeRate: FeeUnit,                     // fee rate per byte
    changeSPK: ScriptPubKey,              // public key
    network: BitcoinNetwork               // bitcoin network information
  ): Future[BitcoinTxBuilder]             // sign TxBuilder to get tx
\end{lstlisting}
Here is how those parameters are generated.

First, a previous transaction with outputs is needed to spend some money from.
This is created here, to show the process.
It could also be parsed from a transaction in the bitcoin network.
A single output is sufficient for this example.
So, first create a new key pair to sign the next transaction and have a scriptPubKey where the money is.
Then define the amount to spend (here 10000 Satoshis) collect this information in a transaction output and add it to the previous transaction.
\begin{lstlisting}[language=scala]
  val privKey = ECPrivateKey.freshPrivateKey
  val creditingSPK = P2PKHScriptPubKey(pubKey = privKey.publicKey)

  val amount = Satoshis(Int64(10000))

  val utxo = TransactionOutput(currencyUnit = amount, scriptPubKey = creditingSPK)

  val prevTx = BaseTransaction(
    version = Int32.one,
    inputs = List.empty,
    outputs = List(utxo),
    lockTime = UInt32.zero
  )
\end{lstlisting}
Next, the new transaction should point to an output of the previous transaction.
Thus, an outpoint is created with the id of the previous transaction and the index pointing to the specific output of it.
This is collected in an utxo spending info which is then put in the list of all utxos (only one here).
\begin{lstlisting}  
  val outPoint = TransactionOutPoint(prevTx.txId, UInt32.zero)

  val utxoSpendingInfo = BitcoinUTXOSpendingInfo(
    outPoint = outPoint,
    output = utxo,
    signers = List(privKey),
    redeemScriptOpt = None,
    scriptWitnessOpt = None,
    hashType = HashType.sigHashAll
  )

  val utxos = List(utxoSpendingInfo)
\end{lstlisting}
Then, the destination, where the money goes, is defined.
This includes a destination script pub key, as well as the amount to spend to it.
\begin{lstlisting}
  val destinationAmount = Satoshis(Int64(5000))

  val destinationSPK = P2PKHScriptPubKey(pubKey = ECPrivateKey.freshPrivateKey.publicKey)

  val destinations = List(
    TransactionOutput(currencyUnit = destinationAmount, scriptPubKey = destinationSPK)
  )
\end{lstlisting}
Finally, define a fee rate, the network params and create a transaction builder.
\begin{lstlisting}
  val feeRate = SatoshisPerByte(Satoshis.one)

  val networkParams = RegTest // soem static values for testing

  val txBuilder: Future[BitcoinTxBuilder] = BitcoinTxBuilder(
    destinations = destinations,
    utxos = utxos,
    feeRate = feeRate,
    changeSPK = creditingSPK,
    network = networkParams
  )
\end{lstlisting}
After calling sign on the transaction builder, a valid transaction is returned.
\begin{lstlisting}
  val signedTxF: Future[Transaction] = txBuilder
    .flatMap(_.sign)
    .map {
      (tx: Transaction) => println(tx.hex) // transaction in hex for the bitcoin network
    }
\end{lstlisting}
There is no need for a transaction input, since a transaction out point is kind of the pointer to a transaction input.

\section{Validation of a Transaction}
Bitcoin-S-Core offers a function called \emph{checkTransaction}.
This is its type signature.
\begin{lstlisting}
  checkTransaction(transaction: Transaction): Boolean
\end{lstlisting}
It takes a transaction as input and returns a Boolean whether the transaction is valid or not.
So for example when passing the built tx from above to this function the returned value would be true.
There are several checks in checkTransaction.
For example, it checks if there is either no input or no output.
In this case it returns false.

The relevant parts for the property to be verified are the following two lines.
\begin{lstlisting}[language=scala]
  val prevOutputTxIds = transaction.inputs.map(_.previousOutput.txId)
  val noDuplicateInputs = prevOutputTxIds.distinct.size == prevOutputTxIds.size
\end{lstlisting}
It gathers all inputs output transaction ids.
On this previous outputs ids it calls distinct and checks if the size stays the same.
Distinct removes duplicates, so if there was two times the same input transaction id the size of those ids would be greater before calling distinct.

There is a bug in those two lines, as described in the next chapter.


\section{Integration}
During this work, Stainless bumped the sbt plugin from version 0.13.17 to 1.2.8 and Scala from 2.11.12 to 2.12.8.
So this section might be out of date now.

Bitcoin-S-Core uses sbt 1.2.8 and Scala 2.12.8, while Stainless sbt plugin is on sbt 0.13.17 and Scala 2.11.12.

Sbt introduced new features in the 1.x release used by Bitcoin-S.

For example, they introduced the new slash syntax.
\begin{lstlisting}[language=scala]
  lazy val root = (project in file("."))
    .settings(
      name := "hello",
      version in ThisBuild := "1.0.0"
    )
\end{lstlisting}

The version line written without slash becomes now one with slash.
The right and left parameter of in-operator reverse and instead of the in-operator is now a slash.
\begin{lstlisting}[language=scala]
  ThisBuild / version := "1.0.0",
\end{lstlisting}

This is just one example of many others.

The bigger problem is, due to the different Scala and sbt versions, the error message, after trying to go in a sbt shell.
\begin{lstlisting}
  [warn] There may be incompatibilities among your library dependencies; run 'evicted'
         to see detailed eviction warnings.
  [error] java.lang.NoClassDefFoundError: sbt/SourcePosition
  ...
  Project loading failed: (r)etry, (q)uit, (l)ast, or (i)gnore?
\end{lstlisting}

Downgrading Bitcoin-S sbt version to 0.13.17 fixes the error but then it can not load some libraries only compiled for newer versions.
So this would take too much time to fix and changes the Bitcoin-S code unwantingly.

The next approach is to use stainless cli instead of sbt.
Running stainless on all source files does not work, because the dependencies are missing.
The parameter \emph{-classpath} can resolve it but the value of this parameter must be the paths to all the dependencies separated by a ':'.
Finally, \emph{core} depends on \emph{secp256k1jni}, another package of Bitcoin-S written in Java.
So this needs to be in the source files to.

The final command looks like this in \emph{core} folder of Bitcoin-S:
\begin{lstlisting}[language=bash]
  $ stainless
    -classpath ".:$(find ~/.ivy2/ -type f -name *.jar | tr '\n' ':')"
    $(find . -type f -name *.scala | tr '\n' ' ')
    $(find ../secp256k1jni -type f -name *.java | tr '\n' ' ')
\end{lstlisting}

The output was some Stainless error.
\begin{lstlisting}
  [Internal] Error: object scala.reflect.macros.internal.macroImpl in compiler mirror
             not found.. Trace:
  [Internal] - scala.reflect.internal.MissingRequirementError$.signal
             (MissingRequirementError.scala:17)
  ...
  [Internal] object scala.reflect.macros.internal.macroImpl in compiler mirror not found.
  [Internal] Please inform the authors of Inox about this message
\end{lstlisting}

After some discussion, we decided to go another way, because the errors may take too much time and it might lead to a next error.
Let's extract the code needed to verify a transaction mainly the class Transaction and ScriptInterpreter with many other classes their depending on.

After this extraction Stainless was successfully integrated with both, sbt and JAR.
This leads us to another finding as described in the next section.


\section{Error Reporting with sbt and JAR}
// TODO https://github.com/epfl-lara/stainless/issues/484?


\section{Fixing a Bug in Bitcoin-S}
After so much digging and trying to understand the checkTransaction code, we found a bug even without Stainless.

Here the relevant code of checkTransaction again:
\begin{lstlisting}[language=scala]
  val prevOutputTxIds = transaction.inputs.map(_.previousOutput.txId)
  val noDuplicateInputs = prevOutputTxIds.distinct.size == prevOutputTxIds.size
\end{lstlisting}

What happens, if a TransactionOutPoint (previousOutput) references the same Transaction ID (txId) another does?

According to the Bitcoin protocol this is possible.
A transaction can have multiple outputs that should be referenceable by the next transaction.
What should not be possible is a transaction referencing the same output twice.
This was a bug in Bitcoin Core known as CVE-2018–17144 which was patched on September 18, 2018.

Here, Bitcoin-S did a bit too much and marked all transaction as invalid, if they referenced a transaction twice in the next transaction.
The fix is, to check on TransactionOutPoint instead of TransactionOutPoint.txId, because TransactionOutPoint contains the txId as well as the output index it references.
\begin{lstlisting}[language=scala]
  val prevOutputs = transaction.inputs.map(_.previousOutput)
  val noDuplicateInputs = prevOutputs.distinct.size == prevOutputs.size
\end{lstlisting}
Since TransactionOutPoint is a case class and Scala has a built in == for case classes there is no need to implement TransactionOutPoint.==.

This was fixed in pull request \mypound435 on GitHub at April 23, 2019 by us along a unit test to prevent this bug from appearing again in the future.

\chapter{Using Bitcoin-S-Core}
\label{chap:using_bitcoins}
This chapter describes the relevant parts needed to verify that a non-coinbase transaction cannot generate new coins in Bitcoin-S-Core.
We will see, how to create valid and invalid transactions with Bitcoin-S-Core and how this transactions are validated against the described property.

\section{Creation of a Transaction}
Some parts of the code in this section are from Bitcoin-S-Core transaction builder example.\cite{BitcoinSCore:txbuilderexample}
Bitcoin-S-Core has a bitcoin transcation builder class with the following signature:
\begin{lstlisting}[language=scala]
  BitcoinTxBuilder(
    destinations: Seq[TransactionOutput], // where we send money
    utxos: BitcoinTxBuilder.UTXOMap,      // unspent transaction outputs
    feeRate: FeeUnit,                     // fee rate per byte
    changeSPK: ScriptPubKey,              // public key
    network: BitcoinNetwork               // bitcoin network information
  ): Future[BitcoinTxBuilder]             // sign TxBuilder to get tx
\end{lstlisting}
Here is how those parameters are generated.

First, a previous transaction with outputs is needed to spend some money from.
This is created here, to show the process.
It could also be parsed from a transaction in the bitcoin network.
A single output is sufficient for this example.
So, first create a new keypair to sign the next transaction and have a scriptPubKey where the money is.
Then define the amount to spend (here 10000 Satishis) collect this information in a transaction output and add it to the previous transaction.
\begin{lstlisting}[language=scala]
  val privKey = ECPrivateKey.freshPrivateKey
  val creditingSPK = P2PKHScriptPubKey(pubKey = privKey.publicKey)

  val amount = Satoshis(Int64(10000))

  val utxo = TransactionOutput(currencyUnit = amount, scriptPubKey = creditingSPK)

  val prevTx = BaseTransaction(
    version = Int32.one,
    inputs = List.empty,
    outputs = List(utxo),
    lockTime = UInt32.zero
  )
\end{lstlisting}
Next, the new transaction should point to an output of the previous transaction.
Thus, a outpoint is created with the id of the previous transaction and the index pointing to the specific output of it.
This is collected in an utxo spending info which is then put in the list of all utxos (only one here).
\begin{lstlisting}  
  val outPoint = TransactionOutPoint(prevTx.txId, UInt32.zero)

  val utxoSpendingInfo = BitcoinUTXOSpendingInfo(
    outPoint = outPoint,
    output = utxo,
    signers = List(privKey),
    redeemScriptOpt = None,
    scriptWitnessOpt = None,
    hashType = HashType.sigHashAll
  )

  val utxos = List(utxoSpendingInfo)
\end{lstlisting}
Then, the destination, where the money goes, is defined.
This includes a destination script pub key, as well as a the amount to spend to it.
\begin{lstlisting}
  val destinationAmount = Satoshis(Int64(5000))

  val destinationSPK = P2PKHScriptPubKey(pubKey = ECPrivateKey.freshPrivateKey.publicKey)

  val destinations = List(
    TransactionOutput(currencyUnit = destinationAmount, scriptPubKey = destinationSPK)
  )
\end{lstlisting}
Finally. define a fee rate, the network params and create a transaction builder.
\begin{lstlisting}
  val feeRate = SatoshisPerByte(Satoshis.one)

  val networkParams = RegTest // soem static values for testing

  val txBuilder: Future[BitcoinTxBuilder] = BitcoinTxBuilder(
    destinations = destinations,
    utxos = utxos,
    feeRate = feeRate,
    changeSPK = creditingSPK,
    network = networkParams
  )
\end{lstlisting}
After calling sign on the transaction builder a valid transaction is returned.
\begin{lstlisting}
  val signedTxF: Future[Transaction] = txBuilder
    .flatMap(_.sign)
    .map {
      (tx: Transaction) => println(tx.hex) // transaction in hex for the bitcoin network
    }
\end{lstlisting}
There is no need for a transaction input, since a transaction out point is kind of the pointer to a transaction input.

\section{Validation of a Transaction (with method checkTransaction)}
Bitcoin-S-Core offers a function called \emph{checkTransaction}.
This is its type signature.
\begin{lstlisting}
  checkTransaction(transaction: Transaction): Boolean
\end{lstlisting}
It takes a transaction as input and returns a boolean whether the transaction is valid or not.
There are several checks in checkTransaction.
For example it checks if there is either no input or no output.
In this case it returns false.

The relevant parts for the property to be verified are the following two lines.
\begin{lstlisting}
  val prevOutputs = transaction.inputs.map(_.previousOutput)
  val noDuplicateInputs = prevOutputs.distinct.size == prevOutputs.size
\end{lstlisting}
It gathers all inputs output.
On this previous outputs it calls distinct and checks if the size stays the same.
Distinct removes duplicates, so if there was two times the same input the size of inputs before calling distinct would be greater.
